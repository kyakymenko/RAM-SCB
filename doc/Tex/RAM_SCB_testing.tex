RAM-SCB comes packaged with a test suite to determine if the current installation operates correctly and produces the expected results. Different tests evaluate different code capabilities and options. Testing is an especially powerful development tool for evaluating the impact, intentional or not, of changes to the source code. A summary of the tests and associated commands can be generated by using {\tt make test\_help}. Current test status can be seen at the github repository.

\section{Using and Interpreting Test Results}

Tests are called through GNU \verb*lmakel. To run all available regression and functional tests, simply use the command
\begin{verbatim}
make test
\end{verbatim}
in the RAM-SCB installation directory. To run a specific test (listed below), call that test by name:
\begin{verbatim}
make test1
\end{verbatim}

\noindent
A test simulation performs the following actions:
\begin{enumerate}
\item{Compile the code with any options required by the particular test.}
\item{Create a run directory entitled \verb*lrun_testl.  If one exists, it will be deleted, so use caution when using this name for any other purpose besides automated testing.}
\item{Copy the required input and parameter files into the new test run directory.}
\item{Run the code. Test simulations are only long enough to perform the features being tested.}
\item{Perform a specialized version of \verb*ldiffl, included in the distribution, to compare the results produced by the test simulation against reference solutions stored in the output directory of the installation location. If the two files have values that differ by a certain amount (default is an absolute difference of $10^{-9}$), the test will fail.}
\end{enumerate}

Each of these steps can be called individually by using {\tt make} \textit{nametest}\_\textit{stepname}, where the step names are {\tt compile}, {\tt rundir}, {\tt run}, and {\tt check}.

If, at any point, the test fails, the details will be recorded in \textit{nametest}.diff in the installation directory. If the test is successful, this file will be created but be empty. Tests that fail at the comparitive stage will list all of the differences. When this happens, additional tools to examine the file differences are recommended. An excellent visualization of file differences can be rendered with \verb*lgvimdiffl, an open source tool that is already installed on many Linux machines.

\section{Description of Available Regression Tests}

\subsection{Test 1}

\textbf{Command:} \verb*lmakel \verb*ltest1l

This is the most basic test for RAM-SCB. It runs RAM-SCB for 900 seconds using Volland-Stern electric field, constant dipole magnetic field (no SCB calculation), and LANL fluxes. \textbf{SHIELDS-RC} users should rely on this test.

\subsection{Test 2}

\textbf{Command:} \verb*lmakel \verb*ltest2l

Test 2 repeats the previous test, but stops half way through the simulation, writes restart files, and restarts the run. This exercises the code's ability to seamlessly restart a simulation at any arbitrary point.

\subsection{Test 3}

\textbf{Command:} \verb*lmakel \verb*ltest3l

Test 3 activates SCB, using a dipole outer boundary condition and the Weimer 2000 empirical electric field. This is the base test for SCB functionality and ionosphere-to-equator electric potential mapping.

\subsection{Test 4}

\textbf{Command:} \verb*lmakel \verb*ltest4l

Test 4 repeats the previous test, but stops half way through the simulation, writes restart files, and restarts the run. Like test2, this exercises the code's ability to seamlessly restart a simulation at any arbitrary point.

\subsection{EMIC Test}

\textbf{Command:} \verb*lmakel \verb*ltestEMICl

This test is a repeat of Test 1 with FLC and EMIC scattering turned on