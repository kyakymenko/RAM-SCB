RAM-SCB output can be visualized in a number of different ways depending on the user's tastes and preferences.  However, a standard library for opening, manipulating, and plotting exists as a sub-module of Spacepy. For 3d visualization of the domain, some basic tools are provided to enable the use of VTK-powered tools such as ParaView.

\section{SpacePy}
Primary support for visualization of RAM-SCB input and output files is provided through the open-source SpacePy library. The latest versioned release of SpacePy can be obtained from the Python Package Index using
\begin{verbatim}
pip install spacepy
\end{verbatim}

Alternatively, the latest stable development version of SpacePy can be obtained from \href{https://github.com/spacepy/spacepy}{the SpacePy git repository}.

For documentation and examples, please see \href{https://spacepy.github.io/autosummary/spacepy.pybats.ram.html}{https://spacepy.github.io/autosummary/spacepy.pybats.ram.html}

\section{VTK: The Visualization Toolkit}
Basic support for VTK-based visualization is provided in the {\tt Scripts/viz} subdirectory.

\subsection{VTK Data from RAM-SCB Restart Files}
`convertRAMrestart.py' batch converts NetCDF restart files to `.vtp' and `.vtu' files

\subsection{Visualization using VTK/ParaView}
Convenience routines to make VTK files with the seed locations are in {\tt makeCustomSource.py}.\\
A standard view of the RAM-SCB domain can be generated using Paraview's scripting capability through the {\tt pvpython} interpreter packaged with ParaView. Running {\tt visualizeRAM.py} with {\tt pvpython} will generate output PNG files showing RAM quantities on the equatorial plane and a cutaway of the SCB field.\\
Other packages such as PyVista, MayaVi, VisIt, etc. can all be used with the VTK files generated by {\tt convertRAMrestart.py}.
