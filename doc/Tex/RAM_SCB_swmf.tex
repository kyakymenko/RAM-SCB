RAM-SCB may be used as a component to the Space Weather Modeling Framework. Under this mode, RAM-SCB receives initial and boundary conditions from the Framework's other components and returns plasma properties to create a two-way coupled system.

Currently the extant documentation for using RAM-SCB as the IM component in SWMF is out of date. For the time being, we recommend that interested users contact the RAM-SCB developers for assistance.
%TODO:

%Update information about getting SWMF (git instead of CSV)
%Talk about how to install RAM-SCB as IM component
%Talk about what currently is coupled
%INSERT TABLE OF COUPLING DESCRIPTIONS.

%\section{Installing RAM-SCB as a SWMF Component}
% QUESTION: Should this entire section just be rewritten? SWMF no longer uses CVS, we have changed the coupling on our end a bit. And SWMF has changed the coupling on their end a lot. Honestly, it may be of more use to have an entirely seperate document for the SWMF coupling as it has its own set of installation instructions and its own set of difficulties.
 
%Obtain a copy of the SWMF and, if necessary, unpack the tarball.  Descend into the Inner Magnetosphere (IM) directory of the Framework.  This is where all inner magnetosphere-type codes used by the Framework are located.  Copy the entire RAM-SCB directory here as {\tt RAM\_SCB}.  It is important to name the directory correctly. CVS users should note that checking out RAM-SCB directly into the SWMF directories will be tricky because of conflicting CVS files located in each Framework directory. Use CVS carefully! 

%Next, move or remove the share directory. RAM-SCB will be using the version obtained through the SWMF. Additionally, making RAM-SCB's share directory unavailable signals RAM-SCB to go into component mode.

%Installing the SWMF and RAM-SCB happens concurrently using the Config.pl script from the top-level directory of the SWMF.  

