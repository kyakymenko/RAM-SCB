Installation and compilation of RAM-SCB requires a Fortran compiler (with OpenMP support), MPI compiled against the chosen Fortran compiler, a Perl interpreter, and several external libraries which have additional requirements (most notably a C compiler, typically standard on Unix-like systems.)  The configuration is done with the Config.pl script; compilation with GNU Make.  The Config/Make system follows the tiered makefile standards of the Space Weather Modeling Framework (SWMF) project; users of the SWMF should feel right at home using RAM-SCB.

\section{Installation of Required Libraries}

There are two required libraries that must be installed to use RAM-SCB: \href{https://www.gnu.org/software/gsl}{GNU Scientific Library} and \href{http://www.unidata.ucar.edu/software/netcdf/}{Unidata's NetCDF library}. It is possible to find pre-compiled binaries for each library, allowing the user to skip the configure, make, and install steps. However, if a binary cannot be found that matches your system and compilers, you will be forced to install from source. Some systems may already have these libraries installed; be sure to check before going through unneccessary installation work.

\subsection{Libraries From Modules}
Well maintained clusters often use the {\tt module} interface for loading and unloading software packages.  Be sure to explore the available packages on a new system before going through the arduous task of installing from source!

\begin{verbatim}
module avail
\end{verbatim}
\noindent
will list all available software libraries on the system.  If a needed software is not available it is a good idea to contact your computing support to see if they can build the module for you before installing from source. You can load a library, unload a library, and list loaded software using the module commands:

\begin{verbatim}
module load package_name
module unload package_name
module list
\end{verbatim}

As an example, here is what you would type on LANL's Grizzly HPC cluster to get started with RAM-SCB:

\begin{verbatim}
module purge
...
\end{verbatim}

The first command purges all currently loaded modules to make sure there is no interference from previously loaded modules. GFortran, MPI, NetCDF and GSL are now loaded and ready to use. Place such commands your shell configuration script ({\tt \verb*l~l/.cshrc} or {\tt \verb*l~l/.bashrc} depending on your shell) to load modules upon login. Note that if you are running using a job scheduler (e.g. slurm, moab, etc...), placing the module commands in your shell configuration script may not be adequate and you they may need to be loaded in your job script.

\subsection{Installation From Source}

If installing from source, follow the instructions provided with the software, making sure to use the same compiler that you are using with RAM-SCB to install the dependencies.

\section{Pointing RAM-SCB to External Libraries}
RAM-SCB must be explicitly pointed to libraries installed in non-standard places. There are several ways for RAM-SCB to find the required external libaries, summarized in Table \ref{tab:libs}. The most permanent and convenient method is to set an environment variable that RAM-SCB will search for upon installation. They can be set with the following commands:

Bash Shell: {\tt export NETCDFDIR=}\textit{installation path}

C Shell: {\tt setenv NETCDFDIR ``}\textit{installation path}\verb*l''l

Placing these commands into configuration files (e.g. {\tt \verb*l~l/.bashrc} or {\tt \verb*l~l/.cshrc}) will load the environment variables at login. This is preferable to typing the commands every session. Note that the installation path listed should point to the top level directory for the installation. For example, {\tt /Users/ram\_user/netcdf} is sufficient but {\tt /Users/ram\_user/netcdf/lib} will cause an error.

Alternatively, it is possible to set the install paths for the external libraries using Config.pl during installation. The use of this method is detailed in the next section. Even if you have set the environment variables, using the Config.pl switches will override them. These switches are \emph{only} recognized in conjunction with the {\tt -install} switch of Config.pl.

Finally, if you set neither the environment variables or the Config.pl switches, the Fortran compiler will look for the libaries in the default directories. This may be useful to test if the libraries are already properly installed by the system administrator.  

\emph{If you change the library locations by any method, you must reinstall the code. These libraries are looked for at runtime, and must be in the same place as at compilation.} 

\begin{table}[ht]
  \centering
  \begin{tabular}{l l l}
  \hline\hline
  Library & Environment Variable & Config.pl Switch\\
  \hline
  GNU Scientific Library & {\tt GSLDIR=[...]} & {\tt -gsl=[...]}\\
  NetCDF & {\tt NETCDFDIR=[...]} & {\tt -netcdf=[...]}\\
  \end{tabular}
\caption{List of required libraries and methods for expressing their location to RAM-SCB. Note that the Config.pl switches override environment variables. If none are given, the Fortran compiler will search in the default library location.}
\label{tab:libs}
\end{table}

\section{Installation, Configuration, and Compiling \label{subchap:install}}
{\tt Config.pl} handles the installation and configuration of RAM-SCB. To view the installation and configuration status, or to view help, use the following commands:
\begin{verbatim}
Config.pl
Config.pl -h
\end{verbatim}

To install the code, use the {\tt -install} flag:

\begin{verbatim}
Config.pl -install
\end{verbatim}


Although RAM-SCB will try to use reasonable defaults based on your system, there are a number of flags that allow you to customize your installation:
\begin{itemize}
\item{Use {\tt -compiler} to select the Fortran 90 compiler. Common choices include gfortran, ifort, and pgf90.}
\item{The {\tt -mpi} flag allows the user to pick which version of MPI to use. Choose from openmpi, mpich, mpich2, and Altix. Alternatively, the {\tt -nompi} flag may be set with no value. This option is fragile.}
\item{Set {\tt -ncdf} to the path of the NetCDF library installation. If used, this flag overrides the environment variable NETCDFDIR.}
\item{Set {\tt -gsl} to the path of the GSL library installation. If used, this flag overrides the environment variable GSLDIR.}
\end{itemize}

To exemplify a typical installation, imagine a machine with several different Fortran compilers available. For each compiler available, the user has a corresponding installation of GSL. The user will use this command to properly install RAM-SCB:
\begin{verbatim}
Config.pl -install -compiler=pgf90 -mpi=mpich2 -gsl=~/libs/gsl_portland/
\end{verbatim}

There are other Config.pl options that set up real precision, debug flags, and optimization level.  Use {\tt Config.pl -h} to learn about the available options.

After the code has been properly configured, compilation is simple:
\begin{verbatim}
make
\end{verbatim}

Compilation is most likely to fail for two reasons. The first is RAM-SCB not finding MPI or another key library.  The second is MPI or an external library that is installed using a mix of different Fortran compilers. Be vigilant when installing each library!

To remove object files before a fresh compilation, use
\begin{verbatim}
make clean
\end{verbatim}

To uninstall RAM-SCB, simply use
\begin{verbatim}
Config.pl -uninstall
\end{verbatim}

\section{Testing the Installation}
Running the RAM-SCB tests is an excellent way to evaluate the success and stability of your installation. To run the tests, simply type
\begin{verbatim}
make test
\end{verbatim}
This will compile RAM-SCB, create a run directory, perform a short simulation and compare the results to a reference solution. If there is a significant difference between the test and reference solution, the test will fail. Details can be found in \verb*Q*.testQ files in the RAM-SCB directory. For a full description of running and interpreting test results, see Chapter 
\ref{chp:test}.

\section{Building Documentation}
To generate a PDF of the latest User Manual, type
\begin{verbatim}
make PDF
\end{verbatim}
The document will be located in the \verb*ldoc/l directory.
